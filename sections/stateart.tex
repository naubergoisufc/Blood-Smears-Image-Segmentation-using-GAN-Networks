\section{Literature Review}

There are a huge number of studies aiming to develop automatic diagnosis systems in the literature. For example, in \cite{tek2006malaria} the authors proposed a method based in color histogram and in \cite{Diaz2009} the authors proposed a method to classify malaria-infected blood smear images. Rajpurkar et al. have developed a reinforcement learning agent (RL) that can predict the probability of an individual positive test for malaria by asking questions about their household \cite{Rajpurkar2017}. The initial studies about the subject do not discuss the necessity of differentiating parasite and non-parasite stained objects which lead to rudimentary solutions. Some studies have addressed the need for parasite detection in order to obtain methods with a better accuracy. Linder et al. \cite{linder2014malaria} have proposed a malaria diagnostic tool for plasmodium falciparum detection. 

There are many different approaches to tackle the problem of building an automatic system to diagnosis malaria, as we focus our work in the use of DCGAN networks and CNN networks, we present a succinct and comprehensive summary of the use of such approaches in the diagnosis of malaria. Again, we reinforce to the careful reader that this is by no means a complete description of the state-of-the-art for automatic diagnosis. 

Convolutional Neural Networks (CNNs) have been  widely applied in many problems of machine learning and computer vision in the last years \cite{krizhevsky2012imagenet,szegedy2015going}. Moreover, a lot of techniques have been proposed to enhance the performance or ease the training of CNNs \cite{simonyan2014very,srivastava2014dropout}. Their popularity began in 2012, following the proposition of the AlexNet network, when it outperformed all other methods for visual classification, thus attracting great attention from the research community \cite{Krizhevsky:2012:ICD:2999134.2999257}. In the following years, an increasing adoption was observed, mainly due to the their ability to have scalable parallel processing through Graphical Processing Units (GPUs), even in usual desktop computers, once they are basically implemented through matrix multiplications easily parallelizable. CNNs have also been successfully employed in a vast domain of applications, such as video object detection, speech recognition and language translation \cite{liu2017survey}.

Recent studies on the deep learning architecture have proven that the pattern classification models based on the deep learning paradigm can significantly outperform the models learned based on conventional classifiers \cite{hinton2002training}, \cite{nair20093d}. Gueorguieva et al. applied the Faster R-CNN object detection model to identify cells and recognize their stages in clear field microscopy images of malaria-infected blood \cite{hung2017applying}. Poostchi et al. presented a comprehensive systematic review with a set of approaches for malaria automatic diagnosis \cite{Poostchi2018}. Zhang et al. presented a two-step approach for detecting infected and uninfected cells. The first step applies an object-detection structure with a trained classifier to detect all red blood cells in a blood drop image. The second stage classifies each segmented region into an infected or uninfected cell, by considering its morphological characteristics \cite{Zhang2016}. Liang et al. employed a convolutional neural network to discriminate infected and uninfected cells in fine blood smears after the application of a conventional approach classification for cell segmentation \cite{Liang2017}.

Other authors which have applied deep learning in cell segmentation are Dong et al. and Gopakumar et al., by means of convolutional neural networks. Dong et al. used whole images of thin blood slides to compile a dataset of red blood cells infected with malaria and uninfected cells, as labeled by a group of four pathologists. The simulation results showed that all these deep convolution neural networks achieved classification accuracies of more than 95 \%, greater than the accuracy of about 92 \% achievable using the support vector machine (SVM) method. In addition, deep learning methods have the advantage of being able to automatically learn the characteristics of the incoming data, thus requiring less amount of human expert inputs for automated malaria diagnosis \cite {Dong2017} \cite{Dong2017a}. Gopakumar et al., in turn, have proposed an image stacking-based approach for automated quantitative malaria detection. The cell counting problem was addressed as a 2-level segmentation strategy and the use of CNN not only improved the detection accuracy but also favored the processing on cell patches and avoided the necessity of hand-engineered features. Slide images have been with a custom-built portable slide scanner made from low-cost, off-the-shelf components \cite{Gopakumar2018ConvolutionalScanner}.

Bibin et al. used deep belief networks and recently Hung et al. presented an end-to-end structure using faster convolutional neural network \cite{Bibin2017} \cite{hung2017applying}. Premaratne et al. worked with digital images of oil immersion views from microscopic slides captured through a capture card. They were preprocessed by segmentation and grayscale conversion to reduce their dimensionality and later fed into a feedforward backpropagation neural network for training \cite{Premaratne2006AFilms}.

Generative adversarial networks (GANs) are deep neural net architectures composed of two networks that pitting one against the other. GANs were introduced by \cite{goodfellow2014generative} in 2014.
GANs have a huge potential to increase the accuracy of the whole method because they can learn to mimic any distribution of data. To the best of our knowledge, these generative adversarial networks have not been applied in the context of malaria detection, which is the main contribution of our study.


%%%%%%%%% texto antigo %%%%%%%%%%%%%%%%%%%%%%%%

%\section{State of Art}

%In this project, a preliminary systematic mapping of the state of the art was conducted based on the process described by Petersen et al. \cite{Petersen2015}, according to which there are five essential steps to be followed: (i) definition of research questions, (ii) conducting research on relevant primary studies, (iii) document screening, (iv) extract keywords of abstracts, and (v) data extraction and mapping. Considering that research questions should exemplify the objectives of the mapping study, the following question was elaborated: Which researches from 2002 to 2018 have applied the use of deep learning in the diagnosis of malaria using peripheral blood smears?

%Initially, the searches have been performed with keywords consisting of the combination of the words \textit{malaria}, \textit{machine learning}, \textit{deep learning} and synonyms. At the first step, all primary studies retrieved were evaluated in order to identify those relevant to answer the research question. After reading titles, abstracts and keywords, this initial set was reduced to 11 articles containing studies related to the use of deep learning. Below we present our main findings.

%Rajpurkar et al. have developed a reinforcement learning agent (RL) that can predict the probability of an individual positive test for malaria by asking questions about their household. After the learning phase, the agent determines the next question in the survey and uses stopping criteria to forecast malaria probability based on their responses so far \cite{Rajpurkar2017}. Gueorguieva et al. applied the Faster R-CNN object detection model to identify cells and recognize their stages in clear field microscopy images of malaria-infected blood \cite{hung2017applying}. Poostchi et al. presented a comprehensive systematic review with a set of approaches for malaria automatic diagnosis \cite{Poostchi2018}.

%Zhang et al. presented a two-step approach for detecting infected and uninfected cells. The first step applies an object-detection structure with a trained classifier to detect all red blood cells in a blood drop image. The second stage classifies each segmented region into an infected or uninfected cell, by considering its morphological characteristics \cite{Zhang2016}. Liang et al. employed a convolutional neural network to discriminate infected and uninfected cells in fine blood smears after the application of a conventional  approach classification for cell segmentation \cite{Liang2017}.

%Other authors which have applied deep learning in cell segmentation are Dong et al. and Gopakumar et al., by means of convolutional neural networks. Dong et al. used whole images of thin blood slides to compile a dataset of red blood cells infected with malaria and uninfected cells, as labeled by a group of four pathologists. Three types of known convolutional neural networks were evaluated, including LeNet, AlexNet and GoogLeNet. The simulation results showed that all these deep convolution neural networks achieved classification accuracies of more than 95 \%, greater than the accuracy of about 92 \% achievable using the support vector machine (SVM) method. In addition, deep learning methods have the advantage of being able to automatically learn the characteristics of the incoming data, thus requiring less amount of human expert inputs for automated malaria diagnosis \cite {Dong2017} \cite{Dong2017a}. Gopakumar et al., in turn, have proposed an image stacking-based approach for automated quantitative malaria detection. The cell counting problem was addressed as a 2-level segmentation strategy and the use of CNN not only improved the detection accuracy but also favored the processing on cell patches and avoided the necessity of hand-engineered features. Slide images have been with a custom-built portable slide scanner made from low-cost, off-the-shelf components \cite{Gopakumar2018ConvolutionalScanner}.

%Bibin et al. used deep belief networks, and recently Hung et al. presented an end-to-end structure using faster convolutional neural network \cite{Bibin2017} \cite{hung2017applying}. Premaratne et al. worked with digital images of oil immersion views from microscopic slides captured though a capture card. They were preprocessed by segmentation and grayscale conversion to reduce their dimensionality and later fed into a feed forward backpropagation neural network for training \cite{Premaratne2006AFilms}.

%With regard to the work reported in this paper, it is innovative in relation to the other studies investigated in the sense that none makes use of Generative Adversarial Networks for the generation of new samples of peripheral blood smears. Through our approach, deep learning models for malaria detection could be fed with a significant test mass of new samples and, thus, improve their accuracy.


