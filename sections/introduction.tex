\section{Introduction}
\label{intro}


The global burden of malaria is enormous. In 2012, the World Health Organization (WHO) estimated that at least 247 million people worldwide suffer from malaria and that more than two billions or 42\% of people worldwide has a malaria contamination risk due to living in malaria-endemic areas, 627,000 of which resulted in deaths among African children. In the Philippines, for example, malaria is considered to be the 9th leading cause of morbidity, with 58 out of the 81 provinces being malaria-endemic. Malaria is a disease caused by a protozoa parasite of the genus plasmodium that infects erythrocytes of patients. One of the plasmodium species that infects humans is Plasmodium falciparum. This species is the most violent because in a short time can invade erythrocytes in large numbers. Moreover, it causes several complications in the body's organs and even causes death. Most of these deaths were caused by the plasmodium falciparum that infects red blood cells of the victims. The patients, in this case, are characterized by a variety of organ dysfunction \cite{Dong2017}. Microscope analysis of blood smear images plays a very important role in the characterization of erythrocytes in the malaria parasites spectrum once that the characteristics of erythrocyte alterations vary accordingly to the malaria parasite responsible for the infection. The microscopic features of the erythrocyte include morphology, intensity and texture \cite{ShuleendaDevi2016}.

Among the major obstacles for malaria eradication is the remote location of the majority of malaria cases and the lack of trained individuals to analyze blood samples using a microscope. The gold standard test for malaria is the method of preparing a blood smear on a glass slide, staining it, and examining it under a microscope. While several fast diagnostic tests are also currently available, they still have disadvantages compared to microscope analysis \cite{Quinn2016DeepDiagnostics}\cite{Premaratne2006AFilms}\cite{Penas2017}. The microscopic analysis of the blood smear by a specialist is a very tedious process and depends on the expertise of the specialist in pathology. This procedure is tedious and erroneous due to the subjectivity in the judgment of the blood smears. Not to mention the lack of trained professions in remote and poor areas deeply affected by malaria. 

Another issue that we may encounter when dealing with malaria diagnosis through microscope analysis is that several laboratories capture the blood smears images using a low-cost microscope system based on a mobile application which produces a low quality blood smears images. Due to the poor quality of the images yield through this system in comparison to traditional light-emitting microscopes, conventional algorithms do not adequately process these images \cite{Sorgedrager2018}.  

To handle such drawbacks, automated systems for diagnosis
steps in. Instead of manually going over a blood sample and
checking for the presence of malaria parasites, photographs
of the sample viewed from the microscope are analyzed by
an intelligent system. With such systems, the early diagnosis of malaria cases in remote location becomes less of a problem, especially if such systems become publicly available to trained microscopists and doctors \cite{Premaratne2006AFilms}\cite{Penas2017}.

In the last decade, deep learning methods have shown successful outcomes in different applications, including signal processing, object recognition, natural language processing, etc. Deep learning can be seen as an extension of well-known multilayer neural network classifiers trained with backpropagation. In a deep learning neural network, we may have several different types of layers that are used to represent linear or non-linear relations between the input and the output of the neural network. Due to the use of various neural layers and, in some cases, complex activation functions, deep learning uses massive amounts of computational power and computational ressources as memory for example. Although time and ressoruce consuming deep learning methods have proven themselves to be the most accurate and reliable methods for several classification problems.

There are several kinds of deep neural networks (neural networks that uses deep learning architecture), each one more successful in a specific sort of problem. Concerning image detection and image recognition, the more suitable deep learning architecture are convolutional neural network (CNN) based algorithms. Those algorithms are more suitable in image related tasks since images have highly correlated intensities in local regions and some local signals or statistics are invariant to location \cite{Yan2017Multi-InstanceRecognition}. The idea of CNN's is to apply smaller convolutional kernels (or filters) in combination with a deep network architecture to capture the discernible image resources as much as possible. 

In the later years, deep learning techniques have boosted the performance of many systems in several areas. Although deep learning is a successful tecnique, it has its shortcomings, the major issue in deep neural networks is that they usually require large training sets. This is the reason why medical applications have been among the latest applications to embrace deep learning, as images are particularly difficult to obtain due to the need for trained experts and privacy issues \cite{Dong2017a}. 

One way to handle the need for large datasets of deep neural networks is through data generation techniques. Generative adversarial networks (GANs) are deep neural net architectures comprised of two nets, pitting one against the other. GANs learn to mimic a distribution of data,  creating new samples in a similar domain. One neural network, called the generator, generates new data instances, while the other, the discriminator, evaluates them for authenticity; i.e. the discriminator decides whether each datum it reviews belongs to the actual training dataset or not  \cite{Goodfellow2014}.  Radford proposes a new model called DCGAN(Deep Convolutional GAN) that uses convolutional layers in a GAN \cite{Radford2015UnsupervisedNetworks}.

In this paper, we aim to use DCGAN networks to generate images of blood smear objects to improve object detection in a malaria diagnosis context. We develop a process for object detection in malaria blood smears using DCGAN networks to train convolutional neural networks. We performed six experiments that showed that the images generated by the DCGAN network can be used in the training of a CNN network to detect objects improving classifier accuracy.

The remain of the paper is organized as follows: in section 2, we present a brief but comprehensive review of the literature concerning the use of DCGAN's and CNN's in general and in disease detection, focusing o malaria detection. In section 3, we present the DCGAN and CNN models proposed and we perform a discussion about the reasons that brought as to select such models. In section 4, we present computational experiments to support the claims made in section 3 concerning the improvement of accuracy in the CNN network through the use of data generated by a DCGAN network. And finally, in section 5, we draw some conclusions driven by the computational results we obtained and lay down some ground to future work and improvement of the proposed method.