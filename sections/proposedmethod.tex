\subsection{Proposed Method}
\label{segmethod}

In this section, we describe the proposed process for object detection in malaria blood smears using CNN networks enhanced by the use of GAN networks to generate more samples and improve the classifier accuracy. The process consists of 5 steps, they are the following steps: 
\begin{enumerate}
\item blood smears image acquisition; 
\item image generation with GANs networks (Figure \ref{fig:maincomp} -\ding{202} ); 
\item train a convolutional neural network (Figure \ref{fig:maincomp} -\ding{203}); 
\item apply adaptive threshold filter (Figure \ref{fig:maincomp} -\ding{204}) and 
\item classify objects with the trained convolutional network (Figure \ref{fig:maincomp} -\ding{205}). 
\end{enumerate}

\begin{figure*}[h]
\caption{Process for object detection in malaria blood smears using DCGAN networks}
\label{fig:maincomp}
  \includegraphics[width=\textwidth]{images/MainComponents.png}
\end{figure*}

The first step is maybe the simplest of the proposed method, the real blood smears images where acquired from the repository presented by Quinn et al. \cite{Quinn2016DeepDiagnostics}. Once we have the image samples we pass to the second step of the proposed method, the image generation by the GAN network. 

We use a deep convolutional generative adversarial network (DCGAN) proposed by Radford et al. in \cite{Radford2015}. The generator receive as input a noise vector which passes through convolutional, normalization, upsampling and activation layers. Batch normalization normalizes activations throughout the network, it prevents small changes to the parameters from amplifying into larger and suboptimal changes in activations in gradients; for instance, it prevents the training from getting stuck in the saturated regimes of nonlinearities. The Relu activation is used in the generator with the exception of the output layer which uses the Tanh function. DCGAN  was trained with 1800 real plasmodium images cutted from real blood smears images.

%
\lstset{basicstyle=\footnotesize\ttfamily,breaklines=true}
\begin{lstlisting}[language=Python,numbers=left,frame=tb,caption=Generator source code,label=lst:gen,xleftmargin=2em]
model = Sequential()

model.add(Dense(512 * 7* 7, activation="relu",     
input_dim=self.latent_dim))

model.add(Reshape((7,7, 512)))
model.add(UpSampling2D())
model.add(Conv2D(256, kernel_size=3, padding="same"))
model.add(BatchNormalization(momentum=0.8))
model.add(Activation("relu"))
model.add(UpSampling2D())
model.add(Conv2D(128, kernel_size=3, padding="same"))
model.add(BatchNormalization(momentum=0.8))
model.add(Activation("relu"))
model.add(UpSampling2D())
model.add(Conv2D(64, kernel_size=3, padding="same"))
model.add(BatchNormalization(momentum=0.8))
model.add(Activation("relu"))
model.add(UpSampling2D())
model.add(Conv2D(32, kernel_size=3, padding="same"))
model.add(BatchNormalization(momentum=0.8))
model.add(Activation("relu"))
model.add(Conv2D(self.channels, 
            kernel_size=3, padding="same"))
model.add(Activation("tanh"))

model.summary()

noise = Input(shape=(self.latent_dim,))
img = model(noise)

\end{lstlisting}

%\input{sections/listing/Gan_list_1.tex}

After the generation of the images by the DCGAN network, we use data augmentation techniques to train a CNN. We follow a simple data augmentation for training: pixels are padded on each side, and a 50x50 crop is randomly sampled from the padded image or its horizontal flip. This allows the network to learn invariance to deformations. Data augmentation is essential to teach the network the desired invariance and robustness properties, when only few training samples are available and realistic deformations can be simulated efficiently.

Finally, we use an adaptive threshold filter in the the blood smear images.  Thresholding is a method of image segmentation where each pixel in an image with a black pixel if the image intensity ${\displaystyle I_{i,j}} I_{{i,j}}$ is less than some fixed constant $T$ (that is, ${\displaystyle I_{i,j}<T} I_{{i,j}}<T)$, or a white pixel if the image intensity is greater than that constant. Adaptive Thresholding is a form of extract useful information encoded into pixels while minimizing background noise. After this procedure, image segments are cut from the blood smears image and classify by the CNN network.