\begin{abstract}

Fast and efficient malaria cases diagnostics are essential in efforts to detect and treat the disease in a proper time. The standard approach to diagnose malaria is a microscope exam, which is submitted to a subjective interpretation. Thus, the automating of the diagnosis process with the use of an intelligent system capable of recognizing malaria parasites could aid in the early treatment of the disease. Usually, laboratories capture a minimum set of images in low quality using a system of microscopes based on mobile devices. Due to the poor quality of such data, conventional algorithms do not process those images properly. This paper presents the application of deep learning techniques to improve the accuracy of malaria plasmodium detection in the previous context.  In order to increase the number of training sets, deep convolutional generative adversarial networks (DCGAN) were used to generate reliable training data that were introduced in our deep learning model to improve accuracy. A total of 6 experiments were performed and a synthesized dataset of 2.200 images was generated by the DCGAN. For a real image database with 600 blood smears with malaria plasmodium, the proposed Deep Learning architecture obtained the accuracy of 100\%. Our results are promising and the solution could be employed to support a mass medical diagnosis.


%Fast and efficient diagnosis of malaria cases is essential in efforts to detect and treat the disease in a proper time. The default approach to diagnose malaria is a microscopy exam, with high level of subjective interpretation. Thus, automating the diagnosis process with the use of an intelligent system capable of recognizing malaria parasites could aid in problem resolution. Commonly, laboratories capture a minimum set of images in low quality using a system of microscopes based on mobile devices. Due to the poor quality of this scheme, conventional algorithms do not process those images properly. This paper presents the application of Deep Learning techniques to improve the accuracy of malaria plasmodium detection.  In order to increase the number of training sets, Deep Convolutional Generative Adversarial Networks (DCGAN) were used to generate reliable training data and improve the accuracy of the proposed model. A total of 6 experiments were performed and a synthesized dataset of 2.200 images was generated by the DCGAN. For a real image database with 600 blood smears with malaria plasmodium, the proposed Deep Learning architecture obtained the accuracy of 100\%. Our results are promising and the solution could be employed to support mass medical diagnosis.


%The use of Deep Learning may be a way to improve the accuracy of the present method. Nevertheless, such an approach usually requires a large number of training sets, which is the reason why is difficult to apply to diagnostic systems deep learning techniques, since the data are usually protected by medical confidentiality. The use of Generating Adverse Networks can help in generating data for training and improving the accuracy of deep learning models. This paper aims at synthesizing malaria blood smears image objects using  Deep Convolutional Generating Adversarial Networks in conjunction with deep learning models in order to improve accuracy in object detection of peripheral blood smears.  Six experiment where performed and  showed that with 2200 images generated by the DCGAN network the classifier obtained a accuracy of 100\% in 600 real images tested.
\keywords{Malaria  \and Generative Adversarial Network \and Deep Learning}
% \PACS{PACS code1 \and PACS code2 \and more}
% \subclass{MSC code1 \and MSC code2 \and more}
\end{abstract}