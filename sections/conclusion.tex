\section{Conclusion}

Malaria is still a serious health problem in many areas of the world and its diagnosis is a key aspect to combat the disease. Microscopic analysis of blood samples is still the preferred method. An expert must examine several blood smears looking for parasites to declare a patient infected or not. 
This study mainly analyzes the use of DCGAN networks in object detection of malaria blood smears. Eight experiment were performed and showed that the images generated by the DCGAN network can be used in the training of a CNN network to detect objects improving classifier accuracy. We used 2200 images generated by DCGAN for training a CNN network which was tested with 600 real images, obtaining 100\% of accuracy.  
For future work, we believe that investigating more sophisticated techniques for detecting objects, such as YOLO and Region Based CNN, will be beneficial. The objects detected by the presented process can be classified allowing a new malaria diagnosis approach.