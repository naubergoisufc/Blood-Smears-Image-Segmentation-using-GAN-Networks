%%%%%%%%%%%%%%%%%%%%%%% file template.tex %%%%%%%%%%%%%%%%%%%%%%%%%
%
% This is a general template file for the LaTeX package SVJour3
% for Springer journals.          Springer Heidelberg 2010/09/16
%
% Copy it to a new file with a new name and use it as the basis
% for your article. Delete % signs as needed.
%
% This template includes a few options for different layouts and
% content for various journals. Please consult a previous issue of
% your journal as needed.
%
%%%%%%%%%%%%%%%%%%%%%%%%%%%%%%%%%%%%%%%%%%%%%%%%%%%%%%%%%%%%%%%%%%%
%
% First comes an example EPS file -- just ignore it and
% proceed on the \documentclass line
% your LaTeX will extract the file if required
\begin{filecontents*}{example.eps}
%!PS-Adobe-3.0 EPSF-3.0
%%BoundingBox: 19 19 221 221
%%CreationDate: Mon Sep 29 1997
%%Creator: programmed by hand (JK)
%%EndComments
gsave
newpath
  20 20 moveto
  20 220 lineto
  220 220 lineto
  220 20 lineto
closepath
2 setlinewidth
gsave
  .4 setgray fill
grestore
stroke
grestore
\end{filecontents*}
%
\RequirePackage{fix-cm}
%
%\documentclass{svjour3}                     % onecolumn (standard format)
%\documentclass[smallcondensed]{svjour3}     % onecolumn (ditto)
%\documentclass[smallextended]{svjour3}       % onecolumn (second format)
\documentclass[twocolumn]{svjour3}          % twocolumn
%
\smartqed  % flush right qed marks, e.g. at end of proof
%
\usepackage{graphicx}
\usepackage[table,xcdraw]{xcolor}
%
% \usepackage{mathptmx}      % use Times fonts if available on your TeX system
%
% insert here the call for the packages your document requires
%\usepackage{latexsym}
% etc.
%
% please place your own definitions here and don't use \def but
% \newcommand{}{}
%
% Insert the name of "your journal" with
% \journalname{myjournal}
%
\begin{document}

\title{Malaria Blood Smears Image Segmentation using GAN Networks}
\subtitle{Do you have a subtitle?\\ If so, write it here}

%\titlerunning{Short form of title}        % if too long for running head

\author{Francisco Nauber Bernardo Gois         \and
        João Alexandre Lôbo Marques \and
        Márcio Costa Santos   \and
        Allberson Bruno de Oliveira Dantas \and 
        %etc.
}

%\authorrunning{Short form of author list} % if too long for running head

\institute{F. Author \at
              first address \\
              Tel.: +123-45-678910\\
              Fax: +123-45-678910\\
              \email{fauthor@example.com}           %  \\
%             \emph{Present address:} of F. Author  %  if needed
           \and
           S. Author \at
              second address
}

\date{Received: date / Accepted: date}
% The correct dates will be entered by the editor


\maketitle

\begin{abstract}
Fast and efficient diagnosis of malaria cases is essential in efforts to eliminate the disease. The default to diagnose malaria is a microscopy exam. This process becomes problematic when cases happen in rural areas and experts cannot be present to make such a diagnosis. Automation of the diagnostical process with the use of an intelligent system that would recognize malaria parasites could aid in problem resolution. Several laboratories capture the images in low quality using a system of microscopes based on mobile devices. Due to the poor quality of this system, conventional algorithms do not process these images pro- perly. The use of Deep Learning may be a way to improve the accuracy of the present method. nevertheless, such an approach usually requires a large number of training sets, which is the reason why is difficult to apply to diagnostic systems deep learning techniques, since the data are usually protected by medical confi- dentiality. The use of Generating Adverse Networks can help in generating data for training and improving the accuracy of deep learning models. The objective of this research project is to use Generating Adversarial Networks in conjunc- tion with deep learning models in order to improve accuracy in the diagnosis of malaria in peripheral blood smears.
\keywords{Malaria  \and Generative Adversarial Network \and More}
% \PACS{PACS code1 \and PACS code2 \and more}
% \subclass{MSC code1 \and MSC code2 \and more}
\end{abstract}

\section{Introduction}
\label{intro}


The global burden of malaria is enormous. In 2012, the World Health Organization (WHO) estimated 207,000,000 cases of malaria globally, 627,000 of which resulted in deaths among African children. In the Philippines, malaria is considered to be the 9th leading cause of morbidity, with 58 out of the 81 provinces being malaria-endemic. Among the major obstacles for malaria eradication are the remote location of the majority of malaria cases and the lack
of trained individuals that can analyze blood samples using
microscopy. This is where automated systems for diagnosis
come in. Instead of manually going over a blood sample and
checking for the presence of malaria parasites, photographs
of the sample viewed from the microscope are analyzed by
an intelligent system. With such systems, the remote location
of malaria cases becomes less of a problem if such systems
become publicly available as trained microscopists and doctors \cite{Premaratne2006AFilms}\cite{Penas2017}.




Malaria rapid diagnostic tests (RDTs) are relatively simple to perform and provide results quickly for making treatment decisions. However, the accuracy and application of RDT
results depends on several factors such as quality of the RDT, storage, transport and end user performance. A cross sectional survey to explore factors that affect the performance and use of
RDTs was conducted in the primary care facilities in South Africa \cite{Moonasar2007}

Our objective was to develop an automated tool recognition of intracellular malaria parasites in stained blood films.    

\section{Deep Learning for Malaria Diagnosis}

In this project, a preliminary systematic mapping of the state of the art was conducted based on the process described by Petersen et al. according to which, there are five essential steps to be followed: (i) definition of research questions, (ii) conducting research on relevant primary studies, (iii) document screening, (iv) keywording of abstracts, and (v) data extraction and mapping.

Premaratne et al. used digital images of oil immersion views from microscopic slides captured though a capture card. They were preprocessed by segmentation and grayscale conversion to reduce their dimensionality and later fed into a feed forward backpropagation neural network (NN) for training \cite{Premaratne2006AFilms}.



\section{The Proposed Method -GAN Segmentation Method}
\label{segmethod}


The GAN network image generation and the blood segmentation process consists of the following components: blood smears image acquisition, image generation with GANs networks,  apply adaptive threshold filter and image segmentation.



% Please add the following required packages to your document preamble:
% \usepackage[table,xcdraw]{xcolor}
% If you use beamer only pass "xcolor=table" option, i.e. \documentclass[xcolor=table]{beamer}
\begin{table}[]
\begin{tabular}{|l|l|l|l|}
\hline
\rowcolor[HTML]{C0C0C0} 
\textbf{\begin{tabular}[c]{@{}l@{}}Number of \\ Samples \\ Created\\ by GAN \\ Network\end{tabular}} & \textbf{\begin{tabular}[c]{@{}l@{}}Number of\\ Real \\ Images\\ for Test\end{tabular}} & \textbf{\begin{tabular}[c]{@{}l@{}}Number \\ of correctly \\ classified \\ samples\end{tabular}} & \textbf{\begin{tabular}[c]{@{}l@{}}Number of \\ incorrectly \\ classified \\ samples\end{tabular}} \\ \hline
600                                                                                            & 600                                                                                 & 420 (70 \%)                                                                                      & 180                                                                                                \\ \hline
800                                                                                            & 600                                                                                 & 406 (68 \%)                                                                                      & 194                                                                                                \\ \hline
1000                                                                                           & 600                                                                                 & 453 (76 \%)                                                                                      & 147                                                                                                \\ \hline
1200                                                                                           & 600                                                                                 & 424 (71 \%)                                                                                      & 176                                                                                                \\ \hline
1400                                                                                           & 600                                                                                 & \begin{tabular}[c]{@{}l@{}}429 \\ ($\sim$71 \%)\end{tabular}                                     & 171                                                                                                \\ \hline
1600                                                                                           & 600                                                                                 & 430 (72 \%)                                                                                      & 170                                                                                                \\ \hline
1800                                                                                           & 600                                                                                 & 480 (80  \%)                                                                                     & 120                                                                                                \\ \hline
2200                                                                                           & 600                                                                                 & 600 (100 \%)                                                                                     & 0                                                                                                  \\ \hline
\end{tabular}
\end{table}




Text with citations \cite{RefB} and \cite{RefJ}.
\subsection{Subsection title}
\label{sec:2}
as required. Don't forget to give each section
and subsection a unique label (see Sect.~\ref{sec:1}).
\paragraph{Paragraph headings} Use paragraph headings as needed.
\begin{equation}
a^2+b^2=c^2
\end{equation}

% For one-column wide figures use
\begin{figure}
% Use the relevant command to insert your figure file.
% For example, with the graphicx package use
  \includegraphics{example.eps}
% figure caption is below the figure
\caption{Please write your figure caption here}
\label{fig:1}       % Give a unique label
\end{figure}
%
% For two-column wide figures use
\begin{figure*}
% Use the relevant command to insert your figure file.
% For example, with the graphicx package use
  \includegraphics[width=0.75\textwidth]{example.eps}
% figure caption is below the figure
\caption{Please write your figure caption here}
\label{fig:2}       % Give a unique label
\end{figure*}
%
% For tables use
\begin{table}
% table caption is above the table
\caption{Please write your table caption here}
\label{tab:1}       % Give a unique label
% For LaTeX tables use
\begin{tabular}{lll}
\hline\noalign{\smallskip}
first & second & third  \\
\noalign{\smallskip}\hline\noalign{\smallskip}
number & number & number \\
number & number & number \\
\noalign{\smallskip}\hline
\end{tabular}
\end{table}


%\begin{acknowledgements}
%If you'd like to thank anyone, place your comments here
%and remove the percent signs.
%\end{acknowledgements}

% BibTeX users please use one of
%\bibliographystyle{spbasic}      % basic style, author-year citations
%\bibliographystyle{spmpsci}      % mathematics and physical sciences
%\bibliographystyle{spphys}       % APS-like style for physics
%\bibliography{}   % name your BibTeX data base

% Non-BibTeX users please use

\bibliographystyle{spmpsci}
\bibliography{mendeley_v2} 



\end{document}
% end of file template.tex

